%
% File acl2013.tex
%
% Contact  navigli@di.uniroma1.it
%%
%% Based on the style files for ACL-2012, which were, in turn,
%% based on the style files for ACL-2011, which were, in turn, 
%% based on the style files for ACL-2010, which were, in turn, 
%% based on the style files for ACL-IJCNLP-2009, which were, in turn,
%% based on the style files for EACL-2009 and IJCNLP-2008...

%% Based on the style files for EACL 2006 by 
%%e.agirre@ehu.es or Sergi.Balari@uab.es
%% and that of ACL 08 by Joakim Nivre and Noah Smith

\documentclass[11pt]{article}
\usepackage{acl2013}
\usepackage{times}
\usepackage{url}
\usepackage{latexsym}
%\setlength\titlebox{6.5cm}    % You can expand the title box if you
% really have to

\title{Diversification of search results by exploiting entity relationships in
web search queries}

\author{Muthu Kumar C.$^{1}$ \hspace{1.1cm}  Senthil Kumar Chandramohan$^{1}$ \hspace{1.1cm} Lee Jia Wei Shaun$^{1}$ \\
 {$^{1}$ National University of Singapore }\\
 {\tt \{a0092669,,\}@nus.edu.sg }%\\\And
 % National University of Singapore \\
 % {\tt email@domain}  \\\And
 % {\tt email@domain} \\
 }

\date{}

\begin{document}
\maketitle
\begin{abstract}
 
\end{abstract}


\section{Introduction}
\label{intro}
Web informational retrieval by commercial search engines focus on 
satisfying information needs of a majority of its users. So, they 
strategize on retrieving documents relevant to the most popular interpretation(s) 
of web search queries. They rely on their dynamic query logs to constantly tune 
their search results to adapt to changes in perceptions of the user. Given 
the fairly mature approaches in this field for retrieving relevant documents, 
recent research has shifted focus on more sophisticated aspects such as addressing 
the implicit information needs of the user by extrapolating from the explicit 
information in the search query. Standard retrieval methods have been extended to 
draw additional information from a user's geographic location \cite{lu2010personalize} 
and personal profiles of the user \cite{teevan2005personalizing} collected by 
encouraging user registration and personalization. Although finding the broad 
geographic location of a user is fairly easy, profiling a user based on his search 
queries requires logging a considerable number of his queries over longer time periods.

To present a diverse list of search results is one alternative when there is lack 
of information specific enough to facilitate retrieval that accurately satisfies the 
users' information needs. Presenting a diverse list of documents as the search result 
of a query enables web search engines to address the information needs of a wider 
audience \cite{bhatia2012analysis}. Inclusion of diversity as a parameter to retrieve 
and rank search results warrants additional considerations. The traditional probabilistic 
ranking principle purely based on relevance would be sub-optimal\cite{gollapudi2009axiomatic}. 
More often than not, diversity in search results require a trade-off against relevance.

\subsection{Types of Diversity}
\label{types}
Although important works in search result diversification such as \cite{agrawal2009diversifying}, 
\cite{santos2010explicit} agree that diversity stems from lack of enough information, 
it is important to realise that variety can be introduced into the query results by 
considering different basis for diversification . \cite{gollapudi2009axiomatic} provide a 
comprehensive set of axioms, only a proper subset of which any diversification algorithm 
would satisfy. Providing search results corresponding to the unambiguous forms of 
an ambiguous query gives semantic diversity. For example, an user issuing a query, 
`Jaguar' could intend to find information on `Jaguar, the wild cat' or `Jaguar, 
the sports car' or `Jaguar the operating system'. Query results on different topics 
of a search query provides topical diversity. For example, an user with the query 
`Barrack Obama', assuming the reference is to the `Barrack Hussain Obama', President 
of the United States of America, could be provided with topical diversity through 
search results on `Barrack Obama on being a Nobel Peace prize winner' or `Barrack 
Obama, an American political leader with African lineage'. In our method, we prioritize 
semantic diversity through disambiguation over topical diversity through categorization.

We note that these various aspects may depend mainly on the interpretations of the 
named entities in the query. Interpretations are governed by ambiguities and 
facets of the named entities. Ambiguities and facets of a named entity depends on 
the level of detail encoded in the query. This leads us to the thought that not 
all queries need diversification and the quantum of diversity would depend on the 
query and the entities in the query.
 
%\cite{santos2010selectively}
%\cite{bhatia2012analysis}

\subsection{Named Entity Recognition in Queries(NERQ)}
\cite{guo2009named} \cite{DBLP:conf/cikm/Pasca07a} claim that a 
significant portion of web search queries contain named entities.
\cite{guo2009named} found that 70\% of queries from the query logs of a 
commercial search engine to contain named entities. Thus recognising named 
entities becomes valuable in interpreting the semantics of a web search 
query. Given such value in recognising named entities, to devise a method 
to recognise them in web search queries becomes important. To recognise
named entity in search queries is different from the traditional 
task of Named Entity Recognition(NER) in Natural Language Processing. 
Traditional NER extracts and tags entities to a pre-defined set of entity 
classes from long coherent textual discourses in documents. The task of 
Named Entity Recognition in Queries(NERQ) is more complicated. \cite{guo2009named} 
found that less than 1\% of their sample of web search queries had two or more 
entities and the average length of a query was 2--3 words. While a standard NER 
today such as \cite{finkel2005incorporating} and \cite{ratinov2009design} rely 
on automatic sequence labelling using Conditional Random Fields(CRF) it fails 
miserably on the much shorter web search queries. \cite{bunescu2006using} show 
a method to extract named entities with the help of encyclopaedia text such as 
Wikipedia pages using heuristics on the full text content of the pages. 
Dbpedia \cite{auer2007dbpedia} is a resource that harvests structured 
content in Wikipedia and provides it in machine-readable form. In our methods we 
extensively use dbpedia to leverage on the crowd-sourced content 
from Wikipedia to recognise named entities and find diversified versions of the 
original search query.

\section{Related works}

\cite{agrawal2009diversifying} describe an algorithm to provide diverse search 
results for ambiguous queries. They assume a taxonomy to build their solution.
\cite{santos2010exploiting} present a framework to identify unspecified information 
needs of underspecified queries to provide a diverse list of search results. 
\cite{cucerzan2007large} uses Wikipedia \cite{wiki:Online} to do large scale entity 
disambiguation. \cite{hoffart2011robust} uses dbpedia among other knowledge bases to 
disambiguate named entities. We propose to build on these works by identifying 
unspecified relations among named entities in a query to diversify the search results. 
In the remainder of this paper we describe our methods in Section~\ref{sec:method}.
%and report on our results in Section~\ref{sec:results}.

\section{Method}
\label{sec:method}
As mentioned earlier in Section~\ref{types}, we start with analysing the user search-
query form ambiguity. We consider the search query as a whole and check for Wikipedia
Disambiguation pages \cite{wikidisamb:Online} through dbpedia's SPARQL end-point 
service. If we find the query considered whole is ambiguous, the reformulations 
of the query are the different disambiguation pages that Wikipedia points to. This part 
handles the semantic ambiguity of the query.

If the query is not ambiguous, we check if it is a category under wikipedia category hierarchy
\cite{wikicat:Online}. If it is a category, then we retrieve all the sub-categories of the 
category as possible reformulations. We interpret that these subtopics represent the different 
sub-topics of the query. This part handles the topical diversity of the search query.

Our method attempts to maximize the utility from the semantics of the complete query before 
analysing its parts. We aim to include elements of generalization to the overall search results 
since we assume that the user is not completely aware of his information need and is rather 
exploring his way through the search. If we fail to detect a match in disambiguation and the 
category pages, we clean the query of prepositions,infinitives and other stopwords and break them 
into many bigrams. We again attempt to match the bigrams to the category hierarchy. If there is a 
match, the reformulation would be to substitute the matched bigram with the {\it sub-categories} 
identified category. For example, the query "Obama family tree" would have the bigrams,
"Obama family", "family tree", "Obama tree". of these, "Obama family" would match the 
category page with the same title. We retrieve the following sub-categories,
\begin{itemize}
\item dbpedia:Category:Barack\_Obama
\item dbpedia:Category:Non-free\_images\_of\_Obama\_family
\end{itemize}
The corresponding reformulations would be to clean the strings and append the token tree with each of 
these category titles. Although the second category title seems less meaningful, we do not filter it 
at this stage and allow the reformulation.

Until now, we have not considered the named entities in the search query. Our consideration of 
named entities is limited to place names and person names. Queries that fail to match any of the above three 
methods would now be checked for name and place entities. We use the dbpedia lookup service \cite{dbpedia:lookup} 
that looks up mirror pages of Wikipedia on dbpedia that stores all the structured content of its peer on Wikipedia.
Again we start by checking the query as a whole for entity. Pages in Wikipedia and their corresponding 
mirror on dbpedia provide class attribute that enables entity class checking. A null value for the class indicates that the page 
does not represent an entity. We filter only those with the value \url{http://dbpedia.org/ontology/Place}
and \url{http://dbpedia.org/ontology/Person}. We then generate bigrams of the search query to check for 
parts of the query that might qualify as an entity. This time the bigrams are limited to those that 
conserve the original word order in the search query. 

We do not attempt to reformulate queries that fail to retrieve reformulations in any of the above methods.
We assume such queries are specific enough to not qualify for diversification. \cite{santos2010selectively} 
argues that  diversification is not suitable for every search query and one should abstain from it when 
not appropriate.


\subsection{Ranking reformulations}
\label{ss:ranking}
We then rank the reformulated search queries using the method prescribed by 
\cite{santos2010exploiting} in equations Eq.8 and Eq.9. They use the metric 
to rank the sub-queries they generate based their relative importance. Since 
our method also generative similar reformulations we reuse this metric.
The metric helps bubble up those reformulations that are meaningful, that is, 
those reformulations that share some of it top 10 search results with the 
original search query result would be ranked at the top. We then select the 
top n reformulations to pick search results to append with the original search 
query result.

\section {Conclusions}
We introduce a novel method to use semantic web data sources with a  
taxonomy of things and identify web search queries or its parts 
in the taxonomy. We then use the retrieved information from the 
taxonomy to reformulate the web search queries to direct the 
information retrieval in multiple directions. These search results 
from a variety of sub-queries can then be combined to produce a 
diverse search result. 

\section*{Acknowledgments}
We thank Dr. NG, Hwee Tou for helping us in choosing our project 
topic and further advise in defining our problem.


\bibliographystyle{acl}
\bibliography{acl2013}

\end{document}
