%
% File acl2013.tex
%
% Contact  navigli@di.uniroma1.it
%%
%% Based on the style files for ACL-2012, which were, in turn,
%% based on the style files for ACL-2011, which were, in turn, 
%% based on the style files for ACL-2010, which were, in turn, 
%% based on the style files for ACL-IJCNLP-2009, which were, in turn,
%% based on the style files for EACL-2009 and IJCNLP-2008...

%% Based on the style files for EACL 2006 by 
%%e.agirre@ehu.es or Sergi.Balari@uab.es
%% and that of ACL 08 by Joakim Nivre and Noah Smith

\documentclass[11pt]{article}
\usepackage{acl2013}
\usepackage{times}
\usepackage{url}
\usepackage{latexsym}
%\setlength\titlebox{6.5cm}    % You can expand the title box if you
% really have to

\title{Diversification of search results by exploiting entity relationships in
web search queries}

\author{Muthu Kumar C.$^{1}$ \hspace{1.1cm}  Senthil Kumar Chandramohan$^{1}$ \hspace{1.1cm} Lee Jia Wei Shaun$^{1}$ \\
 {$^{1}$ National University of Singapore }\\
 {\tt \{a0092669,,\}@nus.edu.sg }%\\\And
 % National University of Singapore \\
 % {\tt email@domain}  \\\And
 % {\tt email@domain} \\
 }

\date{}

\begin{document}
\maketitle
\begin{abstract}
 
\end{abstract}


\section{Introduction}
\label{intro}
Web informational retrieval by commercial search engines focus on 
satisfying information needs of a majority of its users. So, they 
focus on retrieving documents relevant to the most popular interpretation(s) 
of web search queries. They rely on their dynamic query logs to constantly tune their 
search results to adapt to the changes in perceptions of the user. Given the
fairly mature approaches in this field for retrieving relevant documents 
recent research has focussed on more sophisticated aspects such as addressing 
the implicit information needs of the user by extrapolating from the explicit 
information in the search query. Drawing additional information from a user's geographic 
location and personal profiles of the user collected by encouraging user registration 
personalization being the most popular extensions to the standard retrieval methods. 
Although finding the broad geographic location of a user is fairly easy, profiling a
user based on his search queries requires logging a considerable number of his queries 
over longer time periods. 

To present a diverse list of search results is one alternative when there is lack 
of information specific enough to facilitate retrieval that accurately satisfies the 
users' information needs. Presenting a diverse list of documents as the search result 
of a query enables web search engines to address the information needs of a wider 
audience. The quantum of diversity in the search results may depend mainly on the 
interpretations of the named entities in the query.
% citation
Such interpretations are governed by their ambiguities and facets.

\subsection{Types of Diversity}
Although ... agree diversity stems from lack of enough information, variety 
can be introduced into the query results by diversification along different 
lines. Providing search results corresponding to the unambiguous forms of an 
ambiguous query gives semantic diversity \cite{gollapudi2009axiomatic}. For 
example, an user issuing a query, 'Jaguar' could intend to find information on 
'Jaguar, the wild cat' or 'Jaguar, the sports car'. Query results on different 
topics of a search query provides topical diversity. For example, an user with 
a query on 'Barrack Obama', assuming the reference is to the 'Barrack Hussain 
Obama', President of the United States of America, topical diversity could be 
provided by search results on 'Barrack Obama on being a Nobel Peace prize winner' 
or 'Barrack Obama, an American political leader with African lineage'. In our 
method, we prioritize semantic diversity through disambiguation over topical 
diversity through categorization.

 
\cite{santos2010explicit}
\cite{santos2010selectively}
\cite{bhatia2012analysis}

\section{Motivation and Related work}

\cite{agrawal2009diversifying} describe an algorithm to provide diverse search results for
ambiguous queries. \cite{santos2010exploiting} present a framework to identify unspecified
information needs of underspecified queries to provide a diverse list of search results. We
propose to build on these works by identifying unspecified relations among named entities
in a query to diversify the search results.
The remainder of this paper is organized as follows.

\subsection{Named Entity Recognition in Queries(NERQ)}
\cite{guo2009named} \cite{DBLP:conf/cikm/Pasca07a} claim a 
significant portion of web search queries to contain named entities.
\cite{guo2009named} found that 70\% of queries from the query logs of a 
commercial search engine to contain named entities. Thus recognising named 
entities becomes valuable in interpreting the semantics of a web search 
query. Given such value in recognising named entities, to devise a method 
to recognise them in web search queries becomes important. To recognise
named entity in search queries is different from the traditional 
task of Named Entity Recognition(NER) in Natural Language Processing. 
Traditional NER extracts and tags entities to a pre-defined set of entity 
classes from long coherent textual discourses in documents. The task of 
Named Entity Recognition in Queries(NERQ) is more complicated. \cite{guo2009named} 
found that less than 1\% of their sample of web search queries had two or more 
entities and the average length of a query was 2--3 words. While a standard NER 
today such as \cite{finkel2005incorporating} and \cite{ratinov2009design} rely 
on automatic sequence labelling using Conditional Random Fields(CRF) it fails 
miserably on the much shorter web search queries. \cite{bunescu2006using} show 
a method to extract named entities with the help of encyclopaedia text such as 
Wikipedia pages using heuristics on the full text content of the pages.


\section{Method}
Search is verified for ambiguous interpretations \cite{cucerzan2007large} through 

\cite{DBLP:conf/cikm/Pasca07a}

\cite{auer2007dbpedia}
\label{sect:method}
\subsection{Ranking reformulations}
\label{ss:ranking}


\section{Evaluation}
\label{sect:eval}


\section{}
\label{s:layout}

\subsection{Fonts}


\begin{table}[h]
\begin{center}
\begin{tabular}{|l|rl|}
\hline \bf Type of Text & \bf Font Size & \bf Style \\ \hline
paper title & 15 pt & bold \\
author names & 12 pt & bold \\
author affiliation & 12 pt & \\
the word ``Abstract'' & 12 pt & bold \\
section titles & 12 pt & bold \\
document text & 11 pt  &\\
captions & 11 pt & \\
abstract text & 10 pt & \\
bibliography & 10 pt & \\
footnotes & 9 pt & \\
\hline
\end{tabular}
\end{center}
\caption{\label{font-table} Font guide. }
\end{table}

\subsection{The First Page}
\label{ssec:first}

\subsection{Sections}

\subsection{Footnotes}


\subsection{Graphics}

\section{Translation of non-English Terms}

\section{Length of Submission}
\label{sec:length}

\section{Other Issues}


\section*{Acknowledgments}
We thank Dr. Ng,Hwee Thou for helping us in choosing our project topic and further 
advise in defining our problem.


\bibliographystyle{acl}
% you bib file should really go here 
\bibliography{acl2013}


\end{document}
